\section{Conclusions and Future work}
Este estudio presentó la comparación de varias técnicas de machine learning en la predicción de la tasa de desarrollo de la enfermedad llamada Sigatoka negra.  Al respecto, una primera conclusión es que si bien los valores absolutos del $RMSE$ y el $R^2$ difieren entre las dos fincas en estudio, se mostró que al seleccopmar la técnica, la cantidad de semanas de observación, la cantidad de semanas hacia adelante a predecir y las variables a incluir en el modelo, coinciden los resultados. Lo cual es muestra de que estas características estan asociadas al fenómeno más que a una finca en particular. 
%
Además, se pudo mostrar que los menores $RMWE$ were reached with linear models.  
%
Con respecto a cuántas y cuáles variables incluir en el modelo con el fin de obtener un $RMSE$ menor y/o un $R^2$ mayor, se concluye que as little as three meteorological variables can be used because of the correlations detected among variables. 
%
Por otra parte, se concluyó que las ESN obtuvieron $RMWE$ muy altos, lo cual es producto de la cardinalidad del conjunto de datos, que no es suficiente para que la red neuronal se ajuste durante el entrenamiento.
%
Finalmente se destaca que los valores de $R^2$ cercanos al 60\% se alcanzaron en las pruebas.
%
Como trabajo futuro de esta investigación se tiene the extension of these findings to other farms, the extension of the model to be applied in an early warning system, which does not require the regression task, and the use of image to increase the feature sets.


