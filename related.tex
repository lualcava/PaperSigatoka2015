\section{Related works}
\label{sec:related}

Several efforts have been made to apply machine learning techniques in
the automated discovery of relationships between environmental
variables and quantified descriptors for variables of agricultural 
interest such as the progress of diseases.
% 
\citet{Huang2010} summarize in their survey the
development of soft computing techniques in agricultural and
biological engineering, especially in the soil and water context for crop management and decision support in precision agriculture, including fuzzy logic, artificial neural
networks, genetic algorithms, Bayesian inference and decision trees. They do not present numeric results of each paper, only mention the main idea.
%
Similarly, \citet{Kim2014} survey more recent prediction
methods for crop pests using regression and machine learning approaches. Nor do they provide numerical results of each paper.
%
In general, the machine learning methods applied to predict the
evolution of plant diseases, can be classified in two main approaches: 
1)~those whose main inputs are images, and 2)~Those whose main inputs
are environmental and biological variables. Our study focuses in the
second case.
%
\citet{Romero1995} relied on regression models using a stepwise
procedure to predict incubation and disease latency periods for the
black Sigatoka.
% 
He collected environmental data from two different farms in Costa Rica
between December 1993 and August 1995.
%
The prediction models reached coefficients of determination $R^2$ of
69\% or 78\% on the observed data for the incubation and disease
latency periods, respectively; however, the cross validation on
independent data sets failed.
%
In contrast, our proposal presents a model that can be generalizable to other farms that have data.

More recently, \citet{Glezakos2010} used genetic algorithms (GA) and
neural networks (NN) to identify the Tobacco Rattle Virus (TRV) and
the Cucumber Green Mottle Mosaic Virus (CGMMV).
%
The method was tested against some of the most commonly used
classifiers in machine learning (Bayes classifiers, decision trees and
$k$-nearest neighbors) via cross-validation and proved their
applicability in these kind of problems.
%
These authors do not prove their methods in Sigatoka disease and they do classification. Instead we do regression.

\citet{Alves2011} used geoinformation techniques to
develop predictive models in the study of risk areas to soybean rust,
coffee leaf rust, and banana black Sigatoka, under consideration of
Brazil’s climatic characteristics and the distribution of soybean,
coffee and banana crops.
%
Temperature and rainfall data were acquired for the period from 1950
to 2000, and simulated data were generated for 2020, 2050 and 2080
using the SRES~A2 climate change scenarios.
%
Using principal components analysis, a single variable was generated
as a linear combination of 57 input variables, in order to determine
an index explaining 87\%, 88\% and 90\% of the data variability 
of soybean, coffee and 
banana crops, respectively, in municipal districts across Brazil.
%
The climatic model was used to generate the zoning of the three plant
diseases, using temperature and leaf wetness as input.
%
This methodology enabled the visualization of the changes in areas
favorable for epidemics under possible future scenarios of climate
change.
%
How intermediate result, they characterized the monociclic process of the black Sigatoka using  nonlinear regression. Although they do not present the detailed results, it no seem that they wanted to predict the progression of the black Sigatoka in one, two or more periods ahead, how we do.

Other applications of machine learning methods in precision
agriculture include the use of support vector regression to predict
carcass weight in beef cattle in advance to the slaughter
\citep{Alonso2013}, machine learning assessments of soil drying for
agricultural planning \citep{Coopersmith2014}, and early detection and
classification of plant diseases with support vector machines based on
hyperspectral reflectance \citep{Rumpf2010}.

Furthermore, there have been attempts to generate software
tools. \citet{Camargo2012} presented an information
system for the assessment of plant disorders (Isacrodi).
%
They showed that human experts will attain a much accurate assessment
than the Isacrodi classifier, particularly when provided with samples
from the affected crop. However, in those cases where such expertise
is not available, the authors suggest that Isacrodi can still provide
valuable support to farmers.
%
Isacordi includes 15 crop disorders, but the black Sigatoka is none
of them. The prediction process is based on multi-class support vector
machines.

Regarding the prediction of the black Sigatoka disease development
with machine learning methods, \citet{Bendini2013} presented a study
on the risk analysis of its occurrence based on polynomial models.
%
A case study was developed in a commercial banana plantation located
in Jacupiranga, Brazil. It was monitored weekly from February to
December 2005.
%
The data included the weekly monitoring of the disease’s evolution
stage, time series of meteorological data and remote sensing
data.
%
They obtained a model to estimate the evolution of the disease from
satellite imagery. This model relates gray levels (NC) of the band 2
images of the Landsat-5 satellite, with the progress status or disease
severity (EE). The authors claim to reach an $R^2$ of 90\%.

There are also works related to the banana fruit. \citet{Soares2014}
apply two techniques: artificial neural networks (ANNs) and multiple
linear regression (MLR) in banana plant to predict the yield.
%
Their results show that the neural network is more accurate in
forecasting the weight of the bunch in comparison to the multiple
linear regressors in terms of the mean prediction-error $(MPE =
1.40)$, mean square deviation $(MSD = 2.29)$ and coefficient of
determination $(R^2 = 91\%)$.

Although these studies have their contribution, none proposed the kind of preprocessing that we present, nor pose how to predict more than one period ahead without trying to predict climate.
%


