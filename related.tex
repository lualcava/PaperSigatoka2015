
\section{Related works}
\label{sec:related}

Huang et al. \citep{Huang2010}  surveyed the development of soft computing techniques in agricultural and biological engineering, including fuzzy logic, artificial neural networks, genetic algorithms, bayesian inference and decision trees.

A related work, proposed by Romero \citep{Romero1995} relies on regression models using a stepwise procedure to predict incubation and latency times of black Sigatoka. The author performed experiments on two farms located in Costa Rica (La Rita and Waldeck, the same as those used in this study but with different names). The study used data from: December 1993 to August 1995. Romero concluded that the model to predict the incubation period accounted a $R^2$ of 69\% in his observed data but it was not a good predictor when it was validated against an independent dataset (cross validation). For latency, he developed two models that accounted a $R^2$ of 78\% in the observed data, however, when validated against an independent dataset (cross validation), for La Rita obtained an adjusted $R^2$ of 82\%, and for Weldeck, none of the models satisfactorily predicted the latent period; and then those predictions were not shown \citep{Romero1995}.

Glezakos et al. \citep{Glezakos2010} proposed to use Genetic Algorithms (GA) and Neural Networks (NN) to identify plant virus (Tobacco Rattle Virus (TRV) and the Cucumber Green Mottle Mosaic Virus (CGMMV)). This is achieved by the development of ana- lytical tools of evolutionary adaptive width, propelled by Genetic Algorithms (GAs) and Neural Networks (NNs). The method was tested against some of the most commonly used classifiers in machine learning (Bayes, Trees and k-NN) via cross-validation and proved its potential towards the identification. 

In the agricultural context, Alves et al. \citep{Alves2011} used  geoinformation techniques to develop predictive models to study the areas of risk to soybean rust in soybean, coffee leaf rust in coffee, and black Sigatoka in banana, considering Brazil’s climatic characterization and the distribution of soybean, coffee and banana crops. Temperature and rainfall data were obtained for the period from 1950 to 2000, and of simulations for 2020, 2050 and 2080 using the SRES A2 climate change scenarios. Using principal components analysis, a single variable was generated based on 57 variables, in order to determine an index explaining 87\%, 88\% and 90\% of the variability of soybean, coffee and banana crops, respectively, in municipal districts across Brazil. The climatic model was used to generate the zoning of the three plant diseases, using temperature and leaf wetness as input. Areas of favorability for the diseases were plotted against the main coffee, soybean and banana growing areas in Brazil. This methodology enabled the visualization of the changes in areas favorable for epidemics under possible future scenarios of climate change.

Other applications of machine learning methods in precision agriculture include the use of support vector regression to predict carcass weight in beef cattle in advance to the slaughter  \citep{Alonso2013}, machine learning assessments of soil drying for agricultural planning \citep{Coopersmith2014}, and early detection and classification of plant diseases with support vector machines based on hyperspectral reflectance \citep{Rumpf2010}.

Furthermore, there have been attempts to generate software tools. Camargo et al. \citep{Camargo2012} presented an information system for the assessment of plant disorders (Isacrodi). They proposed that experts will attain a much better accuracy than the Isacrodi classifier, particularly when provided with samples from the affected crop. However, those cases where such expertise is not available, they suggest that Isacrodi can provide valuable support to farmers. Isacordi includes 15 crop disorders, but the black Sigatoka no is one of them. The prediction process is based on multi-class support vector machines.

Regarding the prediction of the develpment of the black Sigatoka with machine learning methods, Bendini et al. \citep{Bendini2013}  presented a study about the risk analysis of black Sigatoka occurrence based on polynomial models. A case study was developed in a commercial banana plantation located in Jacupiranga, Brazil. It was monitored weekly during the period from February to December 2005. Data included the weekly monitoring of the disease’s evolution stage, time series of meteorological data and remote sensing data. They obtained a model to estimate the evolution of the disease from satellite imagery. This model relates gray levels (NC) of the band 2 images of the Landsat-5 satellite, with the progress status or disease severity (EE). The authors claim to reach an $R^2$ of 90\%.

Also there are works related to banana fruit. Soares et al. \citep{Soares2014} apply two techniques: artificial neural networks (ANNs) and multiple linear regression (MLR) in banana plant to predict the yield, their results show that the neural network proved to be more accurate in forecasting the weight of the bunch in comparison to the multiple linear regressions in terms of the mean prediction-error $(MPE = 1.40)$, mean square deviation $(MSD = 2.29)$ and coefficient of determination $(R^2 = 91\%)$.

In general, the machine learning methods applied to predict the evolution of plant diseases, can be classified in two main approaches: 1) Those whose main inputs are images, and 2) Those whose main inputs are environmental and biological variables. Our study focuses in the second case.
